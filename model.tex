\documentclass{jpmarticle}

\title{The impact of vector interaction with other insect species on
  disease spread}

\author{
  David Crowder
  \and
  Deborah Finke
  \and
  Jing Li
  \and
  Jan Medlock
  \and
  David Pattemore
  \and
  Rakefet Sharon
}


\newcommand{\elasticity}[2]{\mathcal{E}_{#2}\negthickspace\left({#1}\right)}


\begin{document}

\maketitle

To-do:
\begin{enumerate}
\item Why is the speed in my code sensitive to the spatial step size?
\end{enumerate}


\section{Model description}

Let $r$ be the level of interaction of the vector species with the
other insect species, with $r = 0$ being no interaction.  Initially,
we will consider just this implicit representation of the 2nd insect
species, but later we will explicitly model this population as well.

Let $V_S(x, t)$ and $V_I(x, t)$ be the density per unit distance of
susceptible and infective vectors at distance $x$ and time $t$.
Likewise, let $P_S(x, t)$ and $P_I(x, t)$ be the density per unit
distance of susceptible and infective plants at distance $x$ and time
$t$.  The total density of vectors is
$V(x, t) = V_S(x, t) + V_I(x, t)$ and the total density of plants is
$P(x, t) = P_S(x, t) + P_I(x, t)$.  The model is
\begin{equation}
  \begin{split}
    \frac{\partial V_S}{\partial t}(x, t)
    &= - \beta_V \frac{P_I(x, t)}{P(x, t)} V_S(x, t)
    - \mu_V(r) V_S(x, t)
    \\
    & \quad
    {} + b_V(r) V(x, t) \left[1 - \frac{V(x, t)}{K_V(r)}\right]
    + D_V(r) \frac{\partial^2 V_S}{\partial x^2}(x, t),
    \\
    \frac{\partial V_I}{\partial t}(x, t)
    &= \beta_V \frac{P_I(x, t)}{P(x, t)} V_S(x, t)
    - \mu_V(r) V_I(x, t)
    + D_V(r) \frac{\partial^2 V_I}{\partial x^2}(x, t),
    \\
    \frac{\partial P_S}{\partial t}(x, t)
    &= - \beta_P V_I(x, t) \frac{P_S(x, t)}{P(x, t)},
    \\
    \frac{\partial P_I}{\partial t}(x, t)
    &= \beta_P V_I(x, t) \frac{P_S(x, t)}{P(x, t)}.
  \end{split}
\end{equation}
The total plant density is constant in time, $\frac{\partial
  P}{\partial t}(x, t) = \frac{\partial P_S}{\partial t}(x, t)
+ \frac{\partial P_I}{\partial t}(x, t) = 0$, so
$P(x, t) = P_S(x, t) + P_I(x, t)
\implies P_S(x, t) = P(x, t) - P_I(x, t)$.  Thus
\begin{equation}
  \label{pdesystemdimensional}
  \begin{split}
    \frac{\partial V_S}{\partial t}
    &= - \beta_V \frac{P_I}{P} V_S
    - \mu_V V_S
    + b_V V \left(1 - \frac{V}{K_V}\right)
    + D_V \frac{\partial^2 V_S}{\partial x^2},
    \\
    \frac{\partial V_I}{\partial t}
    &= \beta_V \frac{P_I}{P} V_S
    - \mu_V V_I
    + D_V \frac{\partial^2 V_I}{\partial x^2},
    \\
    \frac{\partial P_I}{\partial t}
    &= \beta_P V_I \left(1 - \frac{P_I}{P}\right).
  \end{split}
\end{equation}

We are initially thinking that the effects of the second insect
species will be
\begin{equation}
  \begin{split}
    \mu_V(r) &= \overline{\mu}_V (1 + a_1 r),
    \\
    b_V(r) &= \overline{b}_V,
    \\
    K_V(r) &= \overline{K_V},
    \\
    D_V(r) &= \overline{D}_V (1 + a_2 r).
  \end{split}
\end{equation}


\section{Nondimensionalization}

The disease-free spatially uniform steady state is
\begin{align}
  V_S = V^* &= K_V \left(1 - \frac{\mu_V}{b_V}\right),
  &
  V_I &= 0,
  &
  P_I &= 0.
\end{align}
We will use $V^*$ for the scale of $V_S$ and $V_I$, $\beta_V$ for the
time scale, and $\sqrt{\frac{\beta_V}{D_V}}$ for the spatial scale:
\begin{equation}
  \begin{aligned}
    \hat{V}_S &= \frac{V_S}{V^*},
    &
    \hat{V}_I &= \frac{V_I}{V^*},
    &
    \hat{P}_I &= \frac{P_I}{P},
    \\
    \hat{t} &= \beta_V t,
    &
    \hat{x} &= \sqrt{\frac{\beta_V}{D_V}} x,
    \\
    \hat{\beta}_P &= \frac{\beta_P}{\beta_V}\frac{V^*}{P},
    &
    \hat{b}_V &= \frac{b_V}{\beta_V},
    &
    \hat{\mu}_V &= \frac{\mu_V}{\beta_V},
  \end{aligned}
\end{equation}
Then
\begin{equation}
  \label{pdesystem}
  \begin{split}
    \frac{\partial \hat{V}_S}{\partial \hat{t}}
    &= - \hat{P}_I \hat{V}_S
    - \hat{\mu}_V \hat{V}_S
    + \hat{b}_V \hat{V} \left[1 - \left(1 - \frac{\hat{\mu}_V}{\hat{b}_V}\right) \hat{V}\right]
    + \frac{\partial^2 \hat{V}_S}{\partial \hat{x}^2},
    \\
    \frac{\partial \hat{V}_I}{\partial \hat{t}}
    &= \hat{P}_I \hat{V}_S
    - \hat{\mu}_V \hat{V}_I
    + \frac{\partial^2 \hat{V}_I}{\partial \hat{x}^2},
    \\
    \frac{\partial \hat{P}_I}{\partial \hat{t}}
    &= \hat{\beta}_P \hat{V}_I (1 - \hat{P}_I).
  \end{split}
\end{equation}

In what follows, we will drop the $\hat{}$, which hopefully won't
cause too much confusion.

\section{Wave-speed analysis}

We assume a solution that is a constant-shape wave traveling at speed
$c$.  The coordinate traveling with this wave is $z = x - c t$.  Let
\begin{equation}
  \begin{split}
    v_S(z) = v_S(x - c t) &= V_S(x, t),
    \\
    v_I(z) = v_I(x - c t) &= V_I(x, t),
    \\
    p_I(z) = p_I(x - c t) &= P_I(x, t).
  \end{split}
\end{equation}
Then \eqref{pdesystem} becomes
\begin{equation}
  \label{nonlinearode}
  \begin{split}
    - c \frac{\md v_S}{\md z}
    &= - p_I v_S - \mu_V v_S
    + b_V v \left[1 - \left(1 - \frac{\mu_V}{b_V}\right) v\right]
    + \frac{\md^2 v_S}{\md z^2},
    \\
    - c \frac{\md v_I}{\md z}
    &= p_I v_S
    - \mu_V v_I
    + \frac{\md^2 v_I}{\md z^2},
    \\
    - c \frac{\md p_I}{\md t}
    &= \beta_P v_I (1 - p_I).
  \end{split}
\end{equation}
Converting this to a first-order system gives
\begin{equation}
  \label{nonlinearode}
  \begin{split}
    \frac{\md v_S}{\md z} &= w_S,
    \\
    \frac{\md w_S}{\md z}
    &= \mu_V v_S
    - b_V v \left[1 - \left(1 - \frac{\mu_V}{b_V}\right) v\right]
    - c w_S + p_I v_S,
    \\
    \frac{\md v_I}{\md z} &= w_I,
    \\
    \frac{\md w_I}{\md z}
    &= \mu_V v_I - c w_I - p_I v_S,
    \\
    \frac{\md p_I}{\md t}
    &= - \frac{\beta_P}{c} v_I (1 - p_I).
  \end{split}
\end{equation}

At the wave front, the susceptible vectors are at carrying capacity
and no infectious vectors or plants,
$v_S = 1$, $v_I = 0$, $p_I = 0$.
Linearizing \eqref{nonlinearode} here gives
% \begin{equation}
%   \label{firstorderode}
%   \begin{split}
%     \frac{\md v_S}{\md z} &= w_S,
%     \\
%     \frac{\md w_S}{\md z}
%     &= 
%     (b_V - \mu_V) v_S - c w_S  + (b_V - 2 \mu_V) v_I + p_I,
%     \\
%     \frac{\md v_I}{\md z} &= w_I,
%     \\
%     \frac{\md w_I}{\md z}
%     &=  \mu_V v_I - c w_I - p_I,
%     \\
%     \frac{\md p_I}{\md t}
%     &= - \frac{\beta_P}{c} v_I,
%   \end{split}
% \end{equation}
% or
\begin{equation}
  \frac{\md \vec{u}}{\md z}
  = \mat{A} \vec{u},
\end{equation}
with
\begin{align}
  \vec{u} &=
  \begin{pmatrix}
    v_S \\ w_S \\ v_I \\ w_I \\ p_I
  \end{pmatrix},
  &
  \mat{A} &=
  \begin{bmatrix}
    0 & 1 & 0 & 0 & 0
    \\
    b_V - \mu_V & - c & b_V - 2 \mu_V & 0 & 1
    \\
    0 & 0 & 0 & 1 & 0 \\
    0 & 0 & \mu_V & - c & - 1 \\
    0 & 0 & - \frac{\beta_P}{c} & 0 & 0
  \end{bmatrix}.
\end{align}

For a traveling wave, we need that all the eigenvalues of $\mat{A}$
are real.  Due to the $0$s in the bottom 3 entries of the left 2
columns, $\mat{A}$ is block diagonal and so its eigenvalues are
\begin{equation}
  \sigma(\mat{A}) = \sigma(\mat{A}_1) \cup \sigma(\mat{A}_2),
\end{equation}
with
\begin{align}
  \mat{A}_1 &=
  \begin{bmatrix}
    0 & 1 \\
    b_V - \mu_V & - c
  \end{bmatrix},
  &
  \mat{A}_2 &=
  \begin{bmatrix}
    0 & 1 & 0 \\
    \mu_V & - c & - 1 \\
    - \frac{\beta_P}{c} & 0 & 0
  \end{bmatrix}.
\end{align}
The eigenvalues of $\mat{A}_1$ are
\begin{equation}
  \lambda = \frac{-c \pm \sqrt{c^2 + 4 (b_V - \mu_V)}}{2},
\end{equation}
which are always real provided that $b_V \geq \mu_V$, which is a
necessary condition for existence of the waves of interest.  The
characteristic equation for $\mat{A}_2$ is
\begin{subequations}
  \begin{equation}
    a_3 \lambda^3 + a_2 \lambda^2 +
    a_1 \lambda + a_0 = 0,
  \end{equation}
  with
  \begin{align}
    a_3 &= c, &
    a_2 &= c^2, &
    a_1 &= - c \mu_V, &
    a_0 &= - \beta_P.
  \end{align}
\end{subequations}
This has all real roots if the discriminant is non-negative:
\begin{equation}
  \begin{split}
    \Delta &=
    18 a_3 a_2 a_1 a_0
    - 4 a_2^3 a_0
    + a_2^2 a_1^2
    - 4 a_3 a_1^3
    - 27 a_3^2 a_0^2
    \\
    &= c^2 \Delta_1,
  \end{split}
\end{equation}
with
\begin{equation}
  \Delta_1
  =
  \left(\mu_V^2 + 4 \beta_P\right) c^4
  + 2 \mu_V \left(2 \mu_V^2 + 9 \beta_P\right) c^2
  - 27 \beta_P^2.
\end{equation}
The discriminant is non-negative if $c = 0$ or $\Delta_1 = 0$.  For
$c = 0$, we have $v_I = 0$, which is the stationary disease-free
equilibrium, which is not the invasion wave of interest.  To find
where $\Delta_1 \geq 0$, we see that $\Delta_1 < 0$ at $c = 0$, so
$\Delta_1 \geq 0$ if
\begin{equation}
  c^2 \geq 
  \frac{2 \left(\mu_V^2 + 3 \beta_P\right)^{3/2}
    - \mu_V \left(2 \mu_V^2 + 9 \beta_P\right)}
  {\mu_V^2 + 4 \beta_P}
  > 0
\end{equation}
which is
\begin{equation}
  \label{speedcondition}
  |c| \geq 
  \sqrt{\frac{2 \left(\mu_V^2 + 3 \beta_P\right)^{3/2}
      - \mu_V \left(2 \mu_V^2 + 9 \beta_P\right)}
    {\mu_V^2 + 4 \beta_P}}.
\end{equation}

The usual theory says that the emergent wave speed is the minimum one
that satisfies \eqref{speedcondition} (although I need to check that the
``cooperativity'' conditions of Weinberger et al are satisfied):
\begin{equation}
  c^* = \pm \sqrt{\frac{2 \left(\mu_V^2 + 3 \beta_P\right)^{3/2}
      - \mu_V \left(2 \mu_V^2 + 9 \beta_P\right)}
    {\mu_V^2 + 4 \beta_P}}.
\end{equation}


% \subsection{Invasion wave of vectors}

% At the wave front, there are no vectors and no infectious plants:
% $v_S = v_I = p_I = 0$.  Linearizing \eqref{nonlinearode} here gives
% \begin{equation}
%   \label{firstorderode0}
%   \begin{split}
%     \frac{\md v_S}{\md z} &= w_S,
%     \\
%     \frac{\md w_S}{\md z}
%     &= (1 - b_V) v_S - c w_S - b_V v_I
%     \\
%     \frac{\md v_I}{\md z} &= w_I,
%     \\
%     \frac{\md w_I}{\md z}
%     &= v_I - c w_I,
%     \\
%     \frac{\md p_I}{\md t}
%     &= - \frac{\beta_P}{c} v_I,
%   \end{split}
% \end{equation}
% or
% \begin{equation}
%   \frac{\md}{\md z}
%   \begin{pmatrix}
%     v_S \\ w_S \\ v_I \\ w_I \\ p_I
%   \end{pmatrix}
%   = \mat{A}
%   \begin{pmatrix}
%     v_S \\ w_S \\ v_I \\ w_I \\ p_I
%   \end{pmatrix},
% \end{equation}
% with
% \begin{equation}
%   \mat{A} =
%   \begin{bmatrix}
%     0 & 1 & 0 & 0 & 0 \\
%     1 - b_V & - c & - b_V & 0 & 0 \\
%     0 & 0 & 0 & 1 & 0 \\
%     0 & 0 & 1 & - c & 0 \\
%     0 & 0 & - \frac{\beta_P}{c} & 0 & 0
%   \end{bmatrix}.
% \end{equation}
% To have a wave, we need all real eigenvalues of $\mat{A}$.  The
% eigenvalues of $\mat{A}$ are
% $\sigma(\mat{A}) = \{0\} \cup \sigma(\mat{A}_1) \cup
% \sigma(\mat{A}_2)$,
% where
% \begin{align}
%   \mat{A}_1 &=
%   \begin{bmatrix}
%     0 & 1 \\
%     1 - b_V & - c
%   \end{bmatrix},
%   &
%   \mat{A}_2 &=
%   \begin{bmatrix}
%     0 & 1 \\
%     1 & - c
%   \end{bmatrix}.
% \end{align}
% These have eigenvalues
% \begin{equation}
%   \begin{split}
%     \sigma(\mat{A}_1) &= \left\{
%       \frac{-c \pm \sqrt{c^2 - 4 (b_v - 1)}}{2}
%     \right\},
%     \\
%     \sigma(\mat{A}_2) &= \left\{
%       \frac{-c \pm \sqrt{c^2 + 4}}{2}
%     \right\}.
%   \end{split}
% \end{equation}
% The eigenvalues of $\mat{A}_2$ are always real.  The eigenvalues of
% $\mat{A}_1$ are real if
% \begin{equation}
%   \label{wavespeed}
%   |c| \geq 2 \sqrt{b_v - 1}.
% \end{equation}
% The minimum wave speed is
% \begin{equation}
%   c^* = \pm 2 \sqrt{b_v - 1}.
% \end{equation}
% This is the usual invasion speed for a single species.



\section{Increasing $r$}

The wave speed in dimensional parameters is
\begin{equation}
  {c^*}^2 = D_V
    \frac{2 \left(\mu_V^2 + 3 \beta V^*\right)^{3/2}
      - \mu_V \left(2 \mu_V^2 + 9 \beta V^*\right)}
    {\mu_V^2 + 4 \beta V^*},
\end{equation}
where
\begin{align}
  V^* &= K_V \left(1 - \frac{\mu_V}{b_V}\right),
  &
  \beta &= \frac{\beta_P \beta_V}{P}.
\end{align}

We are initially thinking that the effects of the second insect
species will be on $\mu_V$ and $D_V$.  Differentiating
\begin{equation}
  \begin{split}
    \frac{\md {c^*}^2}{\md r}
    &=
    \frac{\md}{\md r} \left[
      D_V
      \frac{2 \left(\mu_V^2 + 3 \beta V^*\right)^{3/2}
        - \mu_V \left(2 \mu_V^2 + 9 \beta V^*\right)}
      {\mu_V^2 + 4 \beta V^*}
    \right],
    \\
    &=
    \left(
      \frac{\partial {c^*}^2}{\partial \mu_V}
      +
      \frac{\partial {c^*}^2}{\partial V^*} \frac{\md V^*}{\md \mu_V}
    \right)
    \frac{\md \mu_V}{\md r}
    +
    \frac{{c^*}^2}{D_V} \frac{\md D_V}{\md r},
    \\
    &=
    \left(
      \frac{\partial {c^*}^2}{\partial \mu_V}
      -
      \frac{K_V}{b_V}
      \frac{\partial {c^*}^2}{\partial V^*}
    \right)
    \frac{\md \mu_V}{\md r}
    +
    \frac{{c^*}^2}{D_V} \frac{\md D_V}{\md r},
  \end{split}
\end{equation}
with
\begin{equation}
  \begin{split}
    \frac{\partial {c^*}^2}{\partial \mu_V}
    &=
    D_V
    \frac{\partial}{\partial \mu_V} \left[
      \frac{2 \left(\mu_V^2 + 3 \beta V^*\right)^{3/2}
        - \mu_V \left(2 \mu_V^2 + 9 \beta V^*\right)}
      {\mu_V^2 + 4 \beta V^*}
    \right],
    % \\
    % &=
    % D_V
    % \frac{1}{\mu_V^2 + 4 \beta V^*}
    % \frac{
    %   3 \left(
    %     2 \mu_V \sqrt{\mu_V^2 + 3 \beta V^*}
    %     - 2 \mu_V^2 - 3 \beta V^*
    %   \right)
    %   \left(\mu_V^2 + 4 \beta V^*\right)
    %   -
    %   2 \mu_V
    %   \left[
    %     2 \left(\mu_V^2 + 3 \beta V^*\right)^{3/2}
    %     - \mu_V \left(2 \mu_V^2 + 9 \beta V^*\right)
    %   \right]
    % }
    % {\mu_V^2 + 4 \beta V^*},
    % \\
    % &=
    % {c^*}^2
    % \frac{
    %   3 \left(
    %     2 \mu_V \sqrt{\mu_V^2 + 3 \beta V^*}
    %     - 2 \mu_V^2 - 3 \beta V^*
    %   \right)
    %   \left(\mu_V^2 + 4 \beta V^*\right)
    %   -
    %   2 \mu_V
    %   \left[
    %     2 \left(\mu_V^2 + 3 \beta V^*\right)^{3/2}
    %     - \mu_V \left(2 \mu_V^2 + 9 \beta V^*\right)
    %   \right]
    % }
    % {\left(\mu_V^2 + 4 \beta V^*\right)
    % \left[2 \left(\mu_V^2 + 3 \beta V^*\right)^{3/2}
    %     - \mu_V \left(2 \mu_V^2 + 9 \beta V^*\right)\right]},
    % \\
    % &=
    % {c^*}^2
    % \frac{
    %   2 \mu_V \sqrt{\mu_V^2 + 3 \beta V^*}
    %   \left(\mu_V^2 + 6 \beta V^*\right)
    %   - 3 \left(2 \mu_V^2 + 3 \beta V^*\right)
    %   \left(\mu_V^2 + 4 \beta V^*\right)
    %   + 2 \mu_V^2 \left(2 \mu_V^2 + 9 \beta V^*\right)
    % }
    % {\left(\mu_V^2 + 4 \beta V^*\right)
    % \left[2 \left(\mu_V^2 + 3 \beta V^*\right)^{3/2}
    %     - \mu_V \left(2 \mu_V^2 + 9 \beta V^*\right)\right]},
    \\
    &=
    {c^*}^2
    \frac{
      2 \mu_V \left(\mu_V^2 + 6 \beta V^*\right)
      \sqrt{\mu_V^2 + 3 \beta V^*}
      - \left(
        2 \mu_V^4
        + 15 \beta V^* \mu_V^2
        + 36 \beta^2 {V^*}^2
      \right)
    }
    {\left(\mu_V^2 + 4 \beta V^*\right)
    \left[2 \left(\mu_V^2 + 3 \beta V^*\right)^{3/2}
        - \mu_V \left(2 \mu_V^2 + 9 \beta V^*\right)\right]},
  \end{split}
\end{equation}
and
\begin{equation}
  \begin{split}
    \frac{\partial {c^*}^2}{\partial V^*}
    &=
    D_V
    \frac{\partial}{\partial V^*} \left[
      \frac{2 \left(\mu_V^2 + 3 \beta V^*\right)^{3/2}
        - \mu_V \left(2 \mu_V^2 + 9 \beta V^*\right)}
      {\mu_V^2 + 4 \beta V^*}
    \right],
    \\
    % &=
    % D_V
    % \frac{1}{\mu_V^2 + 4 \beta V^*}
    % \beta
    % \frac{
    %   9 \left(\sqrt{\mu_V^2 + 3 \beta V^*}
    %     - \mu_V\right)
    %   \left(\mu_V^2 + 4 \beta V^*\right)
    %   -
    %   4 \left[2 \left(\mu_V^2 + 3 \beta V^*\right)^{3/2}
    %     - \mu_V \left(2 \mu_V^2 + 9 \beta V^*\right)\right]
    % }
    % {\mu_V^2 + 4 \beta V^*},
    % \\
    % &=
    % {c^*}^2 \beta
    % \frac{
    %   9 \left(\sqrt{\mu_V^2 + 3 \beta V^*}
    %     - \mu_V\right)
    %   \left(\mu_V^2 + 4 \beta V^*\right)
    %   -
    %   4 \left[2 \left(\mu_V^2 + 3 \beta V^*\right)^{3/2}
    %     - \mu_V \left(2 \mu_V^2 + 9 \beta V^*\right)\right]
    % }
    % {\left(\mu_V^2 + 4 \beta V^*\right)
    %   \left[2 \left(\mu_V^2 + 3 \beta V^*\right)^{3/2}
    %     - \mu_V \left(2 \mu_V^2 + 9 \beta V^*\right)\right]},
    % \\
    % &=
    % {c^*}^2 \beta
    % \frac{
    %   9 \sqrt{\mu_V^2 + 3 \beta V^*} \left(\mu_V^2 + 4 \beta V^*\right)
    %   - 8 \sqrt{\mu_V^2 + 3 \beta V^*} \left(\mu_V^2 + 3 \beta V^*\right)
    %   - 9 \mu_V \left(\mu_V^2 + 4 \beta V^*\right)
    %   + 4 \mu_V \left(2 \mu_V^2 + 9 \beta V^*\right)
    % }
    % {\left(\mu_V^2 + 4 \beta V^*\right)
    %   \left[2 \left(\mu_V^2 + 3 \beta V^*\right)^{3/2}
    %     - \mu_V \left(2 \mu_V^2 + 9 \beta V^*\right)\right]},
    % \\
    % &=
    % {c^*}^2 \beta
    % \frac{
    %   \sqrt{\mu_V^2 + 3 \beta V^*} \left[
    %     9 \left(\mu_V^2 + 4 \beta V^*\right)
    %     - 8 \left(\mu_V^2 + 3 \beta V^*\right)
    %   \right]
    %   + \mu_V
    %   \left[4 \left(2 \mu_V^2 + 9 \beta V^*\right)
    %     - 9 \left(\mu_V^2 + 4 \beta V^*\right)
    %   \right]
    % }
    % {\left(\mu_V^2 + 4 \beta V^*\right)
    %   \left[2 \left(\mu_V^2 + 3 \beta V^*\right)^{3/2}
    %     - \mu_V \left(2 \mu_V^2 + 9 \beta V^*\right)\right]},
    % \\
    % &=
    % {c^*}^2 \beta
    % \frac{
    %   \sqrt{\mu_V^2 + 3 \beta V^*} \left(\mu_V^2 + 12 \beta V^*\right)
    %   - \mu_V^3
    % }
    % {\left(\mu_V^2 + 4 \beta V^*\right)
    %   \left[2 \left(\mu_V^2 + 3 \beta V^*\right)^{3/2}
    %     - \mu_V \left(2 \mu_V^2 + 9 \beta V^*\right)\right]},
    \\
    &=
    {c^*}^2 \beta
    \frac{\left(\mu_V^2 + 12 \beta V^*\right)
      \sqrt{\mu_V^2 + 3 \beta V^*} - \mu_V^3}
    {\left(\mu_V^2 + 4 \beta V^*\right)
      \left[2 \left(\mu_V^2 + 3 \beta V^*\right)^{3/2}
        - \mu_V \left(2 \mu_V^2 + 9 \beta V^*\right)\right]}.
  \end{split}
\end{equation}
Alternately, in terms of semi-elasticities,
\begin{equation}
  \elasticity{y}{x}
  = \frac{1}{y} \frac{\partial y}{\partial x}
  = \frac{\partial \log y}{\partial x},
\end{equation}
we have
% \begin{equation}
%   \frac{4}{{c^*}^2}
%   \frac{\md {c^*}^2}{\md r}
%   =
%   \left(
%     \frac{\mu_V}{{c^*}^2}
%     \frac{\partial {c^*}^2}{\partial \mu_V}
%     -
%     \frac{\mu_V}{V^*}
%     \frac{K_V}{b_V}
%     \frac{V^*}{{c^*}^2}
%     \frac{\partial {c^*}^2}{\partial V^*}
%   \right)
%   \frac{r}{\mu_V}
%   \frac{\md \mu_V}{\md r}
%   +
%   \frac{r}{D_V} \frac{\md D_V}{\md r},
% \end{equation}
\begin{equation}
  \elasticity{{c^*}^2}{r}
  =
  \mu_V \left[
    \elasticity{{c^*}^2}{\mu_V}
    - \frac{K_V}{b_V} \elasticity{{c^*}^2}{V^*}
  \right]
  \elasticity{\mu_V}{r}
  +
  \elasticity{D_V}{r},
\end{equation}
with
\begin{equation}
  % \frac{1}{{c^*}^2} \frac{\partial {c^*}^2}{\partial \mu_V}
  % =
  \elasticity{{c^*}^2}{\mu_V}
  =
  \frac{
    2 \mu_V \left(\mu_V^2 + 6 \beta V^*\right)
    \sqrt{\mu_V^2 + 3 \beta V^*}
    - \left(
      2 \mu_V^4
      + 15 \beta V^* \mu_V^2
      + 36 \beta^2 {V^*}^2
    \right)
  }
  {\left(\mu_V^2 + 4 \beta V^*\right)
    \left[2 \left(\mu_V^2 + 3 \beta V^*\right)^{3/2}
      - \mu_V \left(2 \mu_V^2 + 9 \beta V^*\right)\right]},
\end{equation}
and
\begin{equation}
  % \frac{1}{{c^*}^2} \frac{\partial {c^*}^2}{\partial V^*}
  % =
  \elasticity{{c^*}^2}{V^*}
  =
  \beta
  \frac{\left(\mu_V^2 + 12 \beta V^*\right)
    \sqrt{\mu_V^2 + 3 \beta V^*} - \mu_V^3}
  {\left(\mu_V^2 + 4 \beta V^*\right)
    \left[2 \left(\mu_V^2 + 3 \beta V^*\right)^{3/2}
      - \mu_V \left(2 \mu_V^2 + 9 \beta V^*\right)\right]}.
\end{equation}


\section{1 species}

\begin{equation}
  \frac{\partial U}{\partial t}
  = b U \left(1 - \frac{U}{K}\right)
  - \mu U
  + D \frac{\partial^2 U}{\partial x^2}
\end{equation}
  
Traveling wave: $z = x - c t$, $U(x, t) = u(z)$,
\begin{equation}
  - c \frac{\md u}{\md z}
  = b u \left(1 - \frac{u}{K}\right)
  - \mu u
  + D \frac{\md^2 u}{\md z^2}
\end{equation}

Linearizing: $u = 0$
\begin{equation}
  - c \frac{\md u}{\md z}
  =
  (b - \mu) u
  + D \frac{\md^2 u}{\md z^2}
\end{equation}

First order:
\begin{equation}
  \begin{split}
    \frac{\md u}{\md z} &= w
    \\
    \frac{\md w}{\md z}
    &=
    - \frac{b - \mu}{D} u
    - \frac{c}{D} w
  \end{split}
\end{equation}
or
\begin{equation}
  \frac{\md \vec{y}}{\md z}
  =
  \begin{bmatrix}
    0 & 1 \\
    - \frac{b - \mu}{D} & - \frac{c}{D}
  \end{bmatrix}
  \vec{y}
\end{equation}

The characteristic equation is
\begin{equation}
  D \lambda^2 + c \lambda + (b - \mu) = 0
\end{equation}
so
\begin{equation}
  \lambda = \frac{- c \pm \sqrt{c^2 - 4 D (b - \mu)}}{2 D}
\end{equation}
thus
\begin{equation}
  {c^*}^2 = 4 D (b - \mu)
\end{equation}

Now
\begin{equation}
  \begin{split}
    \frac{\md {c^*}^2}{\md r}
    &=
    \frac{\partial {c^*}^2}{\partial D} \frac{\md D}{\md r}
    + \frac{\partial {c^*}^2}{\partial b} \frac{\md b}{\md r}
    + \frac{\partial {c^*}^2}{\partial \mu} \frac{\md \mu}{\md r}
    \\
    &=
    4 (b - \mu) \frac{\md D}{\md r}
    + 4 D \frac{\md b}{\md r}
    - 4 D \frac{\md \mu}{\md r}
    \\
    &=
    {c^*}^2 \left[
      \frac{1}{D} \frac{\md D}{\md r}
      + \frac{b}{b - \mu} \frac{1}{b} \frac{\md b}{\md r}
      - \frac{\mu}{b - \mu} \frac{1}{\mu} \frac{\md \mu}{\md r}
    \right]
  \end{split}
\end{equation}
or
\begin{equation}
  \begin{split}
    \elasticity{{c^*}^2}{r}
    % =
    % \frac{1}{{c^*}^2} \frac{\md {c^*}^2}{\md r}
    % &=
    % D \frac{1}{{c^*}^2} \frac{\partial {c^*}^2}{\partial D}
    % \frac{1}{D} \frac{\md D}{\md r}
    % + b \frac{1}{{c^*}^2} \frac{\partial {c^*}^2}{\partial b}
    % \frac{1}{b} \frac{\md b}{\md r}
    % + \mu \frac{1}{{c^*}^2} \frac{\partial {c^*}^2}{\partial \mu}
    % \frac{1}{\mu} \frac{\md \mu}{\md r}
    % \\
    &=
    D \elasticity{{c^*}^2}{D} \elasticity{D}{r}
    + b \elasticity{{c^*}^2}{b} \elasticity{b}{r}
    + \mu \elasticity{{c^*}^2}{\mu} \elasticity{\mu}{r}
    \\
    &=
    \elasticity{D}{r}
    + \frac{b}{b - \mu} \elasticity{b}{r}
    - \frac{\mu}{b - \mu} \elasticity{\mu}{r}
    \\
    &=
    \elasticity{D}{r}
    + \frac{b \elasticity{b}{r}
      - \mu \elasticity{\mu}{r}}{b - \mu} 
  \end{split}
\end{equation}

For example,
\begin{align}
  \elasticity{b}{r} &= 0,
\end{align}
then
\begin{equation}
  \elasticity{{c^*}^2}{r} 
  =
  \elasticity{D}{r} - \frac{\mu}{b - \mu} \elasticity{\mu}{r}
\end{equation}

\section{Other ideas}

\begin{enumerate}
\item Explicitly model interacting species: $r = X(x, t)$ and then
  some reaction--diffusion equation for $X(x, t)$.
\item The diffusion rate should be density dependent:
  $D_V\big(r, V(x, t)\big)$ or even
  $D_V\big(r, V_S(x, t), V_I(x, t)\big)$.
\item Currently, we have forces of infection
  $\lambda_V = \beta_V \frac{P_I}{P}$ and
  $\lambda_P = \beta_P \frac{V_I}{P}$.  Instead incorporate uninfected
  and infected vector preference for infected and uninfected plants:
  $\lambda_V = \beta_V \frac{a_1 P_I}{P_S + a_1 P_I}$ and
  $\lambda_P = \beta_P \frac{a_2 V_I}{a_2 P_S + P_I}$.
\item The loss of infection in vectors.
\item Exposed classes for vectors and plants.
\item Separate nymph and adult classes of vectors.
\item Separate stationary, feeding, transmitting vectors and moving
  vectors.
\item Separate crawling and flying vectors.
\end{enumerate}

\end{document}
